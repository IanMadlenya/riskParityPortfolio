\documentclass{article}
\usepackage{amsmath, amssymb}
\usepackage{lmodern}
\usepackage{ae}
\usepackage{bm}

\title{Notes on Feng \& Palomar TSP'15 and ICASSP'16: \textit{Risk-parity Portfolio Optimization}}
\author{Z\'e Vin\'icius}
\date{\today}

\begin{document}
\maketitle

\section*{Notation}
Vectors are represented by bold, small-case letters, {\it e.g.}, $\bm{x}$.
All vectors are column vectors. Given a vector $\bm{x}$, both $(\bm{x})_{i}$ and
$x_i$ represent the {\it i}-th component of $\bm{x}$.

\section{Background}

Risky parity is a portfolio design technique which aims to promote
diversification of risk contributions amongst assets. This approach often
leads to a dense portfolio, {\it i.e.}, a portifolio which has contributions from all
its assets.
However, investing in all assets is impractical because of, {\it e.g.}, high transaction
costs.
The problem of jointly designing a sparse risk-parity portfolio is precisely the
subject of study of Feng \& Palomar.

Consider a collection of $n$ assets with random returns
$\bm{r} \in \mathbb{R}^{n}$ such that $\mathbb{E}[\bm{r}] \triangleq \bm{\mu}$ and
$\mathbb{E}\left[(\bm{r} - \bm{\mu})(\bm{r} - \bm{\mu})^{T}\right]$
are its mean vector and its (positive definite) covariance matrix. Also,
let $\bm{w} \in \mathbb{R}^{n}$ denote the normalized portfolio (\textit{e.g.} $\bm{w}^{T}\bm{1} = 1$), which represents the distribution of capital budget
allocated over the assets.

Then, for every normalized portfolio $\bm{w}$, define the portfolio volatility
as $\sigma(\bm{w}) \triangleq \sqrt{\bm{w}^{T}\bm{\Sigma}\bm{w}}$.
Intuitively, the portfolio volativity is a measure of the risk contributions,
i.e., the loss contributions from each asset. Besides, the proper definition
of a measure of risk contribution is a paramount step before actually advancing
on the study of risk parity portfolio.

Note that the portfolio volatility is a positively homogenous function, which
implies that it can be expressed as
\begin{equation}
\sigma(\bm{w}) = \sum_{i=1}^{n}w_{i}\dfrac{\partial \sigma(\bm{w})}{\partial w_i},
\label{eq:eq1}
\end{equation}
in fact, the RHS of (\ref{eq:eq1}) is
\begin{equation}
\sum_{i=1}^{n}w_{i}\dfrac{\partial \sigma(\bm{w})}{\partial w_i} = \sum_{i=1}^{n}
w_i \dfrac{(\bm{\Sigma}{\bm{w}})_{i}}{\sqrt{\bm{w}^{T}\bm{\Sigma}\bm{w}}}.
\label{eq:eq2}
\end{equation}

From~(\ref{eq:eq2}), $w_{i}\dfrac{\partial \sigma(\bm{w})}{\partial w_i}$ can be
thought as the risk contribution of the $i$-th asset. Additionally, the risk
contributions of every asset in a risk-parity portofolio are the same, therefore
\begin{equation}
w_i \dfrac{(\bm{\Sigma}{\bm{w}})_{i}}{\sqrt{\bm{w}^{T}\bm{\Sigma}\bm{w}}} =
w_j \dfrac{(\bm{\Sigma}{\bm{w}})_{j}}{\sqrt{\bm{w}^{T}\bm{\Sigma}\bm{w}}}~\forall i, j.
\end{equation}

\section{Risk-parity portfolio design: problem formulae}
\begin{align}\begin{array}{ll}
    \underset{\bm{w}}{\textsf{minimize}} & \dfrac{1}{2}\bm{w}^{T}\bm{Q}^{k}\bm{w} +
    \bm{w}^{T}\bm{q}^{k} + \lambda F(\bm{w})\\
\textsf{subject to} & \bm{w}^{T}\bm{1} = 1, \bm{w} \in \mathcal{W},
\end{array}\end{align}
where
\begin{align}
    \bm{Q}^{k} & \triangleq 2(\bm{A}^{k})^{T}\bm{A}^{k} + \tau \bm{I},\\
    \bm{q}^{k} & \triangleq 2(\bm{A}^{k})^{T}\bm{g}(\bm{w}^{k}) - \bm{Q}^{k}\bm{w}^{k},
\end{align}
and
\begin{align}
    \bm{A}^{k} & \triangleq \left[\nabla_{\bm{w}} g_{1}\left(\bm{w}^{k}\right), ...,
                                \nabla_{\bm{w}} g_{n}\left(\bm{w}^{k}\right)\right]^{T} \\
    \bm{g}\left(\bm{w}^{k}\right) & \triangleq \left[g_{1}\left(\bm{w}^{k}\right), ...,
                                                   g_{n}\left(\bm{w}^{k}\right)\right]^{T}
\end{align}


\section{Sparse risk-parity portfolio design: problem formulae}
The problem of designing risk-parity portfolios with asset selection, as formulated by Feng \& Palomar,
is given as
\begin{align}\begin{array}{ll}
\underset{\bm{w}, \theta}{\textsf{minimize}} & F(\bm{w}) + \lambda_{1}||\bm{w}||_{0} + \lambda_{2}R(\bm{w}, \theta)\\
\textsf{subject to} & \bm{w}^{T}\bm{1} = 1, \bm{w} \in \mathcal{W},
\end{array}\end{align}
where
\begin{itemize}
    \item $F(\bm{w}) \triangleq \bm{w}^{T}(- \nu \bm{\mu} + \Sigma\bm{w})$
    \item $R(\bm{w}, \theta) \triangleq \sum_{i=1}^{n}(g_i(\bm{w}) - \theta)^{2}\mathbb{I}_{\{w_i \neq 0\}}$
    \item $g_i(\bm{w}) \triangleq w_i\left(\bm{\Sigma}\bm{w}\right)_{i}$
\end{itemize}

\subsection{$\theta$ - update}
For fixed $\bm{w}$, say $\bm{w}^{k}$, the objective function reduces to
\begin{align}\begin{array}{ll}
\underset{\theta}{\textsf{minimize}} & \sum_{i=1}^{n}\left[\left(g_i(\bm{w}^{k}) -
                                       \theta\right)\rho^{\epsilon}_{p}\left(w^{k}_{i}\right)\right]^2,
\end{array}\end{align}
which is the classical univariate weighted least squares problem whose solution is given as
\begin{align}
\hat{\theta} = \sum_{i=1}^{n}x^{k}_{i}g_{i}\left(\bm{w}^{k}\right),
\end{align}
where $x^{k}_{i} = \dfrac{\left(\rho^{\epsilon}_p\left(w^{k}_i\right)\right)^{2}}
                  {\sum^{n}_{j = 1}\left(\rho^{\epsilon}_{p}\left(w^{k}_j\right)\right)^{2}}$.

\subsection{w - update}
For a fixed $\theta$:
\begin{align}\begin{array}{ll}
\underset{\bm{w}}{\textsf{minimize}} & F(\bm{w}) + \lambda_{1}||D^{k}_{o}\bm{w}||^{o}_{o} +
\lambda_{2}P(\bm{w}, \theta^{k}) + \tau ||\bm{w} - \bm{w}^{k}||^{2}_{2}\\
\textsf{subject to} & \bm{w}^{T}\bm{1} = 1, \bm{w} \in \mathcal{W},
\end{array}\end{align}
where
\begin{align}
P(\bm{w}, \theta) \triangleq \sum^{n}_{i=1}\left\{\tilde{g}_{i}(\bm{w}^{k}, \theta) +
                  (\nabla\tilde{g}_{i}(\bm{w}^{k}, \theta))^{T}(\bm{w} - \bm{w}^{k})\right\}^{2}
\end{align}
\begin{itemize}
\item $\tilde{g}_{i}(\bm{w}^{k}, \theta) \triangleq (g_i(\bm{w}^{k}) - \theta) \rho^{\epsilon}_{p}(w^{k}_{i})$
\item $\nabla_{\bm{w}}\tilde{g}_{i}(\bm{w}^{k}, \theta) = \rho^{\epsilon}_{p}(w^{k}_i)\cdot\nabla_{\bm{w}} g_{i}(\bm{w}^{k}) +
       \left[(g_{i}(\bm{w}^k) - \theta)\cdot\nabla_{\bm{w}}\rho^{\epsilon}_{p}(w^k_i)\right]\cdot \bm{e}_i$
\end{itemize}


\end{document}
